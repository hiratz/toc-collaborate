\secput{nphardness}{A proof of {NP}-Hardness}

From the problem formulation, we can easily see that it's a special
case of $k$-clustering problem. If we denote the input $m$ columns
as set $\mathcal{C} = \{c_0, c_1, \ldots, c_{m-1} \}$, the problem
is to partition the input set $\mathcal{C}$ into disjoint subsets
$\mathcal{C}_0, \mathcal{C}_1, \ldots, \mathcal{C}_{k-1}$, such that an
objective function is minimized or maximized. Three of the most commonly
used objective functions include $2$-norm distance (the problem of which
is known as $k$-means), $1$-norm distance (the problem of which is known
as $k$-medians), radius of maximum cluster (the problem of which is known
as $k$-centers). Our problem differs from all these three problems in that
our objective is to minimize the overall summation of {Hamming} distances.

If we denote the $j$-th element of column $c_i$ as $c_i[j]$, the {Hamming}
distance between two input columns $c_i$ and $c_j$ is defined as $H(c_i,
c_j) = |\{l: c_i[l] \neq c_j[l]\}|$~\footnote{notation $|C|$ denotes the 
number of elements of set $C$} and the {Hamming} distance of a
set $\mathcal{C}_i$ is defined as $H(\mathcal{C}_i) = |\{l: \exists
c_{i_1}, c_{i_2} \in \mathcal{C}_i, c_{i_1}[l] \neq c_{i_2}[l]\}|$. We
can further denote the summation of {Hamming} distance of $\mathcal{C}_i$
as $c(\mathcal{C}_i) = |\mathcal{C}_i| \cdot H(\mathcal{C}_i)$, so that
our objective is to minimize $\sum_i c(\mathcal{C}_i)$.

Our problem is actually very similar to the problem of $k$-anonymity,
which is proposed by Sweeney~\cite{Sweeney02} as a technique to release
public information while ensuring both data privacy and data integrity. A
specialization of $k$-anonymity problem asks that given an $g \times m$
($g$ rows, $m$ columns) bit-matrix, each entry of which is either $0$
or $1$, what's the least number of ``x'' (suppression) we need to put on
the bit-matrix to make any column $c_{i_0}$ in the resulting bit-matrix
to be identical to at least $k-1$ other columns $c_{i_1}, c_{i_2},
\ldots, c_{i_{k-1}}$.

By comparing our problem with $k$-anonymity problem, we can see that
the input and objective of these two problems are identical, the only
difference lies in the constraints. Our problem has a constraint that the
number of total clusters can be no more than $k$, while $k$-anonymity
problem has the constraint that for each cluster, the number of total
elements can be no less than $k$. However, if looking carefully, the
constraint of $k$-anonymity problem can be rephrased as

\begin{enumerate}
\item number of elements per cluster be no less than $k$.

\item total number of clusters be no more than $m/k$, which is a 
constraint implied from the first constraint.
\end{enumerate}

So, we can see that the only difference of our problem from $k$-anonymity
problem is that $k$-anonymity problem has an additional constraint
on the number of elements per cluster. That's it! Regarding the similarity
of our problem to the $k$-anonymity problem, we re-name our problem to 
$k$-distinct problem.

Since the $k$-anonymity problem has already been proved {NP}-hard by
numerous ways \cite{}, we will try performing a similar reduction from
a known {NP}-hard problem, $k$-dimensional matching, to our $k$-distinct
problem. Note that a naive reduction from $k$-anonymity to $k$-distinct
won't work because the global optimum of one instance can not guarantee
a global optimum of the other on either direction.

{\bf $k$-dimensional perfect matching:} Given a collection $\mathcal{C}$
of $k$-sets over a universe $\mathcal{U}$, is there a subset $\mathcal{S}
\subseteq \mathcal{C}$ such that:

\begin{itemize}
\item Every $x \in \mathcal{U}$ is in some $k$-set $s$ of $\mathcal{S}$
\item The sets of $\mathcal{S}$ are disjoint; i.e. $\forall s_1, 
s_2 \in \mathcal{S}$, $s_1 \cap s_2 = \emptyset$.
\end{itemize}

Note that when $k = 2$, the $k$-dimensional perfect matching problem 
is polynomial time solvable but is {NP}-hard for $k \geq 3$.

\begin{theorem}
For $k \leq m/3$, the $k$-distinct problem is {NP}-hard.
\end{theorem}

\begin{IEEEproof}
Given an arbitrary instance of $k$-dimensional matching problem, where
$\mathcal{U} = \{x_0, x_1, \ldots, x_{m-1}\}$, $\mathcal{C} = \{s_0, s_1,
\ldots, s_{g-1}\}$ such that $\forall j \in [0, g-1]$, $s_j \subseteq
\mathcal{C}$ and $|s_j| = k$, we construct an $\frac{m}{k}$-distinct
problem as follows:

\begin{itemize}
\item rows correspond to $s_j \in \mathcal{C}$.
\item columns correspond to $x_i \in \mathcal{U}$.
\end{itemize}

For any entry of bit-matrix 
\begin{equation*}
T[j, i] = \left \{ \begin{array}{lr}
                   1 & \mbox{if $x_i \in s_j$} \\
                   0 & \mbox{otherwise}
                   \end{array} \right.
\end{equation*}

An example of this construction:

Suppose a $3$-dimensional matching problem: $\mathcal{U} = \{1, 2, 3,
4, 5, 6\}$ and $\mathcal{C} = \{\{1, 2, 3\}, \{1, 4, 5\}, \{4, 5, 6\},
\{2, 3, 6\}\}$. The constructed bit-matrix is as follows:

\begin{table}[!ht]
\centering
\begin{tabular}{|c|c|c|c|c|c|c|}
\hline 
              & $1$ & $2$ & $3$ & $4$ & $5$ & $6$ \\
\hline
$\{1, 2, 3\}$ & $1$ & $1$ & $1$ & $0$ & $0$ & $0$ \\
\hline
$\{1, 4, 5\}$ & $1$ & $0$ & $0$ & $1$ & $1$ & $0$ \\
\hline
$\{4, 5, 6\}$ & $0$ & $0$ & $0$ & $1$ & $1$ & $1$ \\
\hline
$\{2, 3, 6\}$ & $0$ & $1$ & $1$ & $0$ & $0$ & $1$ \\
\hline
\end{tabular}
\caption{Example of reducing $k$-dimensional matching problem 
to $\frac{m}{k}$-distinct problem}
\label{tab:egKRed}
\end{table}

We claim that $k$-dimensional matching problem has a perfect matching ,
i.e. there is a subset $\mathcal{S} \subseteq \mathcal{C}$ of size $m/k$
such that each $x_i \in \mathcal{U}$ belongs to one and only one $s_j
\in \mathcal{S}$ $\iff$ the constructed $\frac{m}{k}$-distinct problem
has an optimal solution of grouping all columns into $\frac{m}{k}$
clusters and has a cost (the number of ``x'' put in bit-matrix) less
than or equal to $g m - \frac{m^2}{k} - \max\{0, \frac{m}{k} - k\} \cdot
(g k - m) = gk^2 - mk$, if we assume $m/k - k \geq 0$.

$\Longrightarrow : $ If we have a $k$-dimensional perfect matching, i.e.
a subset $\mathcal{S} \subseteq \mathcal{C}$ of size $m/k$ such that each
$x_i \in \mathcal{U}$ belongs to one and only one $s_j \in \mathcal{S}$.
Apparently, $\forall s_j, s_i \in \mathcal{S}$, $|s_j| = |s_i| = k$
and $s_i \cap s_j = \emptyset$. So if we group all columns $x_i \in s_j$
together, no entries in row $j$ will be covered by ``x''. For the $m/k$
rows that corresponds to the sets in perfect matching, there
are $m/k \cdot m = m^2/k$ entries that are not covered by ``x''. For
the rest $g - m/k$ rows that corresponds to the sets not in perfect
matching, each row has at least $\max\{0, m/k - k\}$ clusters that are not
``x'' out, each these cluster has $k$ entries, so that the total entries
in the rest $g - m/k$ rows that are not ``x'' out is $\max\{0, (m/k -
k)\} \cdot k \cdot (g - m/k) = \max\{0, m/k - k\} \cdot (gk - m)$.  So,
the overall cost (number of ``x'') of constructed $\frac{m}{k}$-distinct
problem won't be larger than $g \cdot m - m^2/k - \max\{0, \frac{m}{k}
- k\} \cdot (g k - m) = gk^2 - mk$.

$\Longleftarrow : $ If we can find a $\frac{m}{k}$-clustering of all
columns in the constructed bit-matrix with a cost no more than $gk^2 -
mk$, let's prove that the solution corresponds to a $k$-dimensional
perfect matching of original problem.

Because $gk^2 - mk = gm - m^2/k - (m/k - k) \cdot k \cdot (g - m/k)$,
apparently, it means that at least $m/k$ rows are entirely uncovered 
by ``x''. For the rest $g - m/k$ rows, it has $(m/k - k) \cdot k$ in 
total entries that are not covered by ``x''.
HOW TO PROCEED ALONG THIS DIRECTION???
\end{IEEEproof}

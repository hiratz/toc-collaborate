\secput{intro}{Introduction}

We consider the \defn{polygonal partial-sum problem} in which a static
$2$-dimensional grid $A$ with $N$ entries chosen from a
commutative semigroup $(S, \oplus)$ is given and can be preprocessed
for subsequent querying.  Each query designates a polytopal region $q$
of grid coordinates, and the problem is to compute the ``sum'' of the
points in the region:
%
\begin{equation} 
\sum_{(k_1,\ldots,k_d) \in q} A[k_1,\ldots,k_d]\ .
\label{eq:defIrregularRangeQuery}
\end{equation}
The polygonal query is restricted in that the normals to the sides
of the polygon must be known at
preprocessing time.  Polygonal query regions generalize the assumption
common in the literature of orthotope-shaped queries (boxes).  For
example, in two dimensions, a query region could be triangular with
sides of known slope or even a complex polygon.

Since $(S, \oplus)$ is a commutative semigroup, $\oplus$ is an
associative and commutative operator.  No additional restrictions are
imposed on this semigroup as is sometimes done for similar problems in
the literature:
\begin{itemize}

\item We do not assume idempotence ($a \oplus a = a$ for $a \in S$).

\item We do not assume any partial or total ordering of the elements
  of~$S$.

\item We do not assume the existence of an inverse operation
  (subtraction), since otherwise a trivial $\Theta(N)$-time,
  $\Theta(N)$-space preprocessing algorithm for solving this problem
  exists~\cite{Yao82}.
\end{itemize} 

We use the \defn{arithmetic model} for complexity analysis, the same
complexity model used by A. Yao \cite{Yao82, Yao85} and Chazelle and
Rosenberg \cite{ChazelleRo89, ChazelleRo91}.  In this model, the space
bound of the preprocessing algorithm is counted in units of semigroup
elements.  The preprocessing time and query time count the number of
arithmetic (semigroup) operations performed, ignoring the time needed
to find the proper memory cells~\cite{Yao82}\footnote{Actually, it's
a simple LCA (Lowest Common Ancestor) problem for complete binary tree, 
which can be done by using bit-tricks in $O(1)$ time}.

Many results have been reported in the literature on $1$-D and
multidimensional orthogonal range-query problems, the special case
of the polytopal partial-sum problem when the query is an orthotope.
Yao's work \cite{Yao82, Yao85} on $1$-D orthogonal queries laid down
the foundation for later research on general partial-sums in semigroups.
Chazelle and Rosenberg \cite{ChazelleRo89} extended Yao's $1$D algorithm
to multidimensional grids using the method of \defn{dimension reduction}.
The key idea behind the dimension reduction method for a $d$-dimensional
orthogonal partial-sums query is as follows.  During preprocessing of
a $d$-dimensional grid, pick one grid dimension $i \in [1, d]$, and
decompose the grid into a $1$-D sequence of $(d-1)$-dimensional grids.
Apply Yao's $1$-D algorithm on the $1$-D sequence by treating each
$(d-1)$-dimensional grid as a semigroup element and applying $\oplus$
on them component-wise.  The same procedure is then applied recursively
on each $(d-1)$-dimensional grid.  During the query stage, a query range
is decomposed into a small number of slices by considering the problem
as a $1$-D grid of $(d-1)$-dimensional grids.  This process produces
a set of $(d-1)$-dimensional subproblems, each of which can be solved
recursively.  For the base case, the $1$-D case can be solved using a
``two-path'' recursion --- one solving ``small'' queries (intrablock
queries), and the other solving ``large'' queries (interblock queries). In
\cite{ChazelleRo91}, they further proved that the asymptotic bound of
Yao's $1$-D algorithm is optimal.

To the best of our knowledge, all previous research assumes orthogonal
query ranges.  During the development of the Pochoir stencil compiler
\cite{TangChKu11, TangChLu11}, however, we encountered the problem of
querying octagonally shaped ranges in a $2$-D grid.  Our study of that
problem led us to this research.

In summary, our contributions are as follows:
\begin{itemize}

\item We extend the concept of range queries in a multidimensional
  grid from orthotopes to polytopes, the faces of which have normals
  known during the preprocessing stage.

\item We introduce the notion of a \defn{meta-algorithm} to handle
  range queries over a multi-dimensional grid, including orthogonal
  range queries in arbitrary $d$-dimensional grids and polygonal range
  queries in $2$-dimensionals. Compared with dimension reduction method,
  the \defn{meta-algorithm} treats the input algorithm as a black box,
  doesn't need to know any internal data structures created or used by
  the input algorithm, which has the advantage that any initial algorithm
  that just does the work can be plugged into the \defn{meta-algorithm}
  and possibly hit the optimal asymptotic bound by undergoing several
  rounds of self-application. The key idea of \defn{meta-algorithm} is
  established on the observation that all algorithms operate on data,
  by simply manipulating or compressing the input data to the algorithm,
  we can achieve the effect of reducing the asymptotic bound without
  any apriori knowledge of the input algorithm.
  %\celnote{Compare with dimension reduction.}

\item We show that the sum of all the grid points within a given
%  $k$-faceted polytope in a $d$-dimensional 
  $k$-sided polygon in a $2$-dimensional 
  grid with $N$ grid points
  can be computed in the same asymptotic bound as 
% orthotope range.
  orthogonal range. 
  \punt{i.e. $\Theta(k\alpha^d(N))$ time using $\Theta(kN)$
  preprocessing space and time, where $\alpha(N)$ is a functional
  inverse of Ackermann's function.}

% \item We parallelize the preprocessing stage, achieving a parallelism
%   of $\Theta(kN/\log N)$ in the work-span model
%   \cite[Ch.~27]{CormenLeRi09}, assuming $k=O(N)$.

\item We implemented the meta-algorithm for $1$-D and $2$-D
  grids. To the best of our knowledge, this is the first experimental
  study and performance results of static partial sum algorithm 
  with $\alpha$ bound in general semi-group, where $\alpha(n)$ stands
  for the functional inverse of Ackermann's function. 
  \punt{and parallelized it using Intel Cilk
  Plus~\cite{IntelCilkPlus10}.\celnote{This contribution needs to be
  quantified, compared with other implementations, or dropped.}
  \yuannote{To the best of my knowledge, there is no previous
  implementation on partial sum in semi-group, since Chazelle and
  Rosenberg's algorithm is very tricky / hard to implement, or in some
  sense, un-implementable.  But there is some implementation on RMQ. But
  first it's a different problem with different bound; secondly, if
  I recall correctly, their O(N) bound algorithm sometimes is slower
  than some naive algorithm due to the heavy internal overhead which is
  not counted in the simple arithmetic model. But our implementation,
  as you will see in \secref{expr}, the performance numbers match well
  with their theoretical bounds. }} % end punt
\end{itemize}

The main constraint of our current approach is that the normals of all
sides of the query polygon must be given beforehand.  Whether
polygonal partial-sum queries can be efficiently answered when not all
slopes are known in advance is left as an open problem. As for how to
extend the methodology of processing $2$-D polygonal range queries to
even higher-dimensional polytopal range queries is another open problem.

\punt{
The main constraint of our current approach is that the normals of all
facets of the query polytope must be given beforehand.  Whether
polytopal partial-sum queries can be efficiently answered when not all
slopes are known in advance is left as an open problem.
}
% punt ends

% organization of the paper
The rest of the paper is organized as follows.  \secref{meta}
elaborates on the notion of meta-algorithm and illustrates its use to
answer orthogonal range queries in $d$-dimensional grids.
\secref{poly} explains how to answer polytopal queries using the
meta-algorithm.  \secref{expr} presents some experimental results for
$1$D, $2$D, and $3$D grids.
%  of bounds from $N \log^d (N)$, $N (\log^*
% (N))^d$, $\ldots$, up to $N \alpha(N)^d$, parallelized them by Intel
% Cilk Plus. 
% Finally, we offer some concluding remarks in \secref{future}.  

% LocalWords:  semigroup orthotope idempotence preprocessing Yao 
% LocalWords:  online semigroups subproblems intrablock interblock 
% LocalWords:  octagonally orthotopes polyhedra polytope Ackermann's
% LocalWords:  parallelize parallelized
% LocalWords:  Chazelle Yao's Pochoir Cilk polytopal

Motivation and Application of Range 1-bit Query problem:

Range query has wide applications in database systems. There are
roughly two classes of range query, dynamic range query and static range
query. Dynamic range query allows dynamic INSERT and / or DELETE elements,
while the static version allows a preprocessing before answering any
online query.  This paper studies only static range query problem and
adopts the arithmetic model which only counts the number of arithmetic
operation and ignores the operations of memory access.

If the arithmetic operation is general partial sum, a near-linear
algorithm of $(O(N \alpha N), O(\alpha N))$ ((preprocessing bound,
query bound) for $1$-D array is known \cite{Yao82}. Chazelle and
Rosenberg~\cite{ChazelleRo89} have extended it to multi-dimensional
array with similar near-linear bound. If the arithmetic operation is
$\min$ or $\max$, by exploiting Four-Russian trick, one can further
reduce the preprocessing bound to linear $O(N)$. However, no previous
results or work have ever explored the possibility of sub-linear space
in some special case.  A sub-linear space preprocessing algorithm does
not only reduce the preprocessing complexity, it also reduce possible
cache complexity simply by using less memory.

This paper discusses such a special case ---- {R1Q}, i.e. Range 1-bit
Query problem, with the arithmetic operation to be bit-wise and / or.
We not only explored the orthogonal (rectangular) {R1Q}, but also
non-orthogonal versions.  For the {R1Q} problem, even the exact algorithm
can be sub-linear in $1$-D case. If we can tolerate some errors, the
space complexity of the preprocessing algorithm can be further reduced.
